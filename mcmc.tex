\documentclass[12pt,twoside,pdftex]{article}
\usepackage{styles/dar_endnotes}

\newcommand{\this}{Using Markov Chain Monte Carlo}
\usepackage{amssymb}
\usepackage{amsmath}
\usepackage{mathrsfs}
\usepackage{natbib}
\usepackage{float}
\usepackage{graphicx}

% Figure captions
\makeatletter
\newsavebox{\tempbox}
\newcommand{\@makefigcaption}[2]{%
\vspace{-10pt}{#1.--- #2\par}}%
\renewcommand{\figure}{\let\@makecaption\@makefigcaption\@float{figure}}
\makeatother

% No overflow
\setlength{\emergencystretch}{2em}

% Typeography
\newcommand{\documentname}{\textsl{Chapter}}
\newcommand{\sectionname}{Section}
\newcommand{\equationname}{equation}
\newcommand{\project}[1]{\textsl{#1}}
\newcommand{\foreign}[1]{\textsl{#1}}
\newcommand{\code}[1]{\texttt{\detokenize{#1}}}
\newenvironment{pseudocode}{\begin{ttfamily}\detokenize\obeylines}{}
\newcommand{\aposteriori}{\foreign{a~posteriori}}
\newcommand{\apriori}{\foreign{a~priori}}
\newcommand{\adhoc}{\foreign{ad~hoc}}
\newcommand{\etal}{\foreign{et al.}}
\newcommand{\eg}{\foreign{e.g.}}
\newcommand{\vs}{\foreign{vs.}}
\newcommand{\affil}[1]{{\footnotesize\textsl{#1}}}
\newcommand{\acronym}[1]{{\small{#1}}}

% Notes
\newcommand{\notename}{note}
\newcommand{\note}[1]{\endnote{#1}}
\def\enotesize{\normalsize}
\renewcommand{\thefootnote}{\fnsymbol{footnote}} % the ONE footnote needs this

% Problems
\newcommand{\problemname}{Problem}
\newcounter{problem}
\newenvironment{problem}{\paragraph{\problemname~\theproblem:}%
\refstepcounter{problem}}{}

% Math stuff
\newcommand{\mmatrix}[1]{\boldsymbol{#1}}
\newcommand{\inverse}[1]{{#1}^{-1}}
\newcommand{\transpose}[1]{{#1}^{\scriptscriptstyle \top}}
\newcommand{\parametervector}[1]{\mmatrix{#1}}
\newcommand{\bvec}[1]{\mmatrix{#1}}
\newcommand{\setof}[1]{\{{#1}\}}
\newcommand{\dd}{\mathrm{d}}
\newcommand{\given}{\,|\,}
\newcommand{\mean}[1]{\left<{#1}\right>}

% Headers
\renewcommand{\MakeUppercase}[1]{#1}
\pagestyle{myheadings}
\renewcommand{\sectionmark}[1]{\markright{\thesection.~#1}}
\markboth{\chaptertitle}{}

% References.
\usepackage{color,hyperref}
\definecolor{linkcolor}{rgb}{0,0,0.5}
\hypersetup{colorlinks=true,linkcolor=linkcolor,citecolor=linkcolor,
            filecolor=linkcolor,urlcolor=linkcolor}
\newcommand{\arxiv}[1]{\href{http://arxiv.org/abs/#1}{\textsl{arXiv}:#1}}
\newcommand{\doi}[1]{\href{http://doi.org/#1}{\textsc{doi}:#1}}
\newcommand{\isbn}[1]{\textsc{isbn:}#1}


\begin{document}

\thispagestyle{plain}\raggedbottom
\section*{Data analysis recipes:\\ \this\footnotemark}

\footnotetext{%
    The notes begin on page~\pageref{note:first}, including the
    license\note{\label{note:first}
        Copyright 2010 by the authors. This work is licensed under a
        \href{http://creativecommons.org/licenses/by-nc-nd/3.0/deed.en\_US}{%
            Creative Commons Attribution-NonCommercial-NoDerivs 3.0 Unported
            License}.}
    and the acknowledgements\note{%
        We would like to thank
          Jo Bovy (IAS),
          Brendon Brewer (Auckland),
          Jonathan Goodman (NYU),
          Dustin Lang (CMU),
          \ldots
        for valuable advice and comments.}.
}

\noindent
Dan~Foreman-Mackey\\
\affil{Center~for~Cosmology~and~Particle~Physics, Department~of~Physics,%
       New York University}
\\[1ex]
David~W.~Hogg\\
\affil{Center~for~Cosmology~and~Particle~Physics, Department~of~Physics,%
       New York University}\\
\affil{Max-Planck-Institut f\"ur Astronomie, Heidelberg}

\begin{abstract}
    WTF is MCMC?
\end{abstract}

\section{When do you need MCMC?}

...Search, optimization, and sampling in inference.

\section{What is MCMC?}

\section{Metropolis--Hastings, stretch move, and more}

...M-H, emcee, and references to competitor methods like nested and hamiltionian etc.

\section{Results, error bars, and figures}

...using the output to make figures, get numbers with error bars, and show results

\section{Tuning}

...tuning and Gibbs sampling and the like

\section{Convergence}

...how do you know you have run long enough?

\section{Initialization and burn-in}

\section{Advice and discussion}

...Don'ts and no-nos and breaking Markov-ness.

% \begin{figure}[htbp]
% \exampleplot{ex17}
% \caption{Partial solution to \problemname~\ref{prob:bayesintrinsic}:
% The marginalized posterior probability distribution for the intrinsic
% variance.}\label{fig:bayesintrinsic}
% \end{figure}

\clearpage
\markright{Notes}\theendnotes

\clearpage
\begin{thebibliography}{}\markright{References}
\bibitem[Foreman-Mackey et al.(2012)]{emcee}
  Foreman-Mackey,~D., Hogg,~D.~W., Lang,~D., \& Goodman,~J.\ 2012,
  \project{emcee}: The MCMC Hammer,
  \arxiv{1202.3665}
\end{thebibliography}

\end{document}
